\documentclass{acm}
\usepackage[dutch]{babel}

\begin{document}

\title{User Experience Design}
\subtitle{Muziektafel}

\numberofauthors{5}
\author{
\alignauthor Robert Dahmen\\
       \affaddr{Universiteit Twente}\\
       \email{r.dahmen@me.com}
\alignauthor Alexander Drechsel
       \affaddr{Universiteit Twente}\\
       \email{alexander\_drechsel3@ \\ hotmail.com}
\and
\alignauthor Niels Kamp\\
       \affaddr{Universiteit Twente}\\
       \email{niels.kamp@xs4all.nl}
\alignauthor Kasper Vaessen\\
       \affaddr{Universiteit Twente}\\
       \email{k.m.vaessen@ \\ student.utwente.nl}
\alignauthor Niels Visser\\
       \affaddr{Universiteit Twente}\\
       \email{nsvisser@gmail.com}
}
\maketitle

\begin{abstract}
Samenvatting.
\end{abstract}

\keywords{multi-touch, table, inexperienced user}

\section{Introductie}
Multi-touch interfaces zijn de laatste jaren steeds meer onderdeel geworden van het dagelijks leven; smartphones, tablets hebben hier allemaal ondersteuning voor en maken levendig gebruik van de mogelijkheden die multi-touch biedt. Maar naast deze apparaten die steeds meer mensen bij zich hebben, zijn er ook ontwikkelingen op multi-touch gebied voor apparaten met grotere schermen. Denk hierbij aan computermonitoren, televisies en zogenaamde video walls: de technologische ontwikkeling richt zich op multi-touch.

Zo ook bij opdrachtgever KITT Engineering, waar men tracht een brug te slaan tussen creatieve idee\"en en de uiteindelijke toepassing daarvan. KITT heeft in-house een multi-touch tafel ontwikkeld, waarmee het de mogelijke toepassingen van grote multi-touch oppervlakken wil onderzoeken. Ook wil men bekijken hoe de bedachte technologie in de echte wereld gebruikt kan worden en hoe mensen reageren op dergelijke technologie wanneer ze ermee in aanraking komen.

\subsection{Opdrachtsomschrijving}
De opdracht bestaat uit het ontwikkelen van een applicatie die de mogelijkheden van multi-touch ten volle benut en mensen op een intu\"itieve manier kennis laat maken met het fenomeen van een multi-touch interface ter grootte van een tafel. De applicatie moet mensen zowel uitnodigen om zelfstandig uit te proberen wat een multi-touch tafel kan als de mogelijkheid bieden om gezamenlijk te werken aan het cre\"eren van iets moois.

Hoe aan deze opdracht invulling gegeven moet worden, is niet gespecificeerd. Het is aan de projectgroep om deze invulling te bedenken en de opdrachtgever ten behoeve van sturing bij het ontwerpproces te betrekken.

\subsection{Inhoud}
In dit paper wordt in hoofdstuk~\ref{sec_domein} ingegaan op het domein van de te realiseren applicatie: door wie zal het eindresultaat gebruikt worden en met welk doel. Ook wordt beschreven aan welke functionele eisen de gerealiseerde applicatie moet voldoen om de doelgroep succesvol dit doel te laten bereiken. In hoofdstuk~\ref{sec_concept} staat beschreven hoe deze functionele eisen verwerkt zijn in meerdere conceptuele ontwerpen, welke inspiratiebronnen zijn gebruikt voor de bedachte concepten en welk van deze ontwerpen vervolgens in samenspraak met de opdrachtgever is uitgekozen om verder uit te werken. Dit verder uitwerken wordt gedaan in hoofdstuk~\ref{sec_detail}. In dit hoofdstuk staat een functionele specificatie van het gekozen ontwerp en een beschrijving van de technische realisatie: een technisch ontwerp.

Na het ontwerpen en realiseren van de applicatie, zal de usability ervan getest moeten worden, alsook de user experience. De opzet van deze tests en de resultaten ervan staan beschreven in hoofdstuk~\ref{sec_evaluatie}. Naar aanleiding van deze resultaten zal in hoofdstuk~\ref{sec_discussie} behandeld worden in hoeverre de gerealiseerde applicatie voldoet aan de eerder gestelde eisen.

Tot slot volgt in hoofdstuk~\ref{sec_conclusie} een reflectie van het doorlopen ontwerpproces, het eindresultaat en eventuele verbeterpunten.

\section{Domein}
\label{sec_domein}
Een multi-touch tafel is een experimentele techniek die nog niet breed toegepast wordt in de wereld. Het wordt meestal voor specialistische toepassingen gebruikt, zoals het bedrijfsleven, en in state-of-the-art producten, die bedoeld zijn om de mogelijke toepassingen te onderzoeken -- zoals het gebruik in musea om te kijken of kinderen met een interactief systeem meer opsteken van hun bezoek of hun bezoek positiever beoordelen.

Het concept multi-touch is met de sterke opkomst van multi-touch capable smartphones en tablets echter niet meer uit de hedendaagse wereld weg te denken. Wel is het zo dat de adoptie van deze apparaten voornamelijk in de jongere demografen een groeispurt doormaakt, terwijl de ouderen van dagen hier minder in meegaan. Aangezien het doel van de tafel is om iedereen, jong en oud, kennis te laten maken met het concept van een multi-touch tafel, is het van belang om bij het ontwerp rekening te houden met de intu\"itiviteit van het gebruik ervan: iedereen moet ermee kunnen werken. Eventueel na een korte instructie, maar bij voorkeur zonder.

Vanwege het open karakter van de opdracht -- het mensen laten kennismaken met het concept multi-touch tafel door middel van een applicatie die uitnodigt tot gebruik -- zijn de functionele eisen in deze fase beperkt. Vanwege de brede doelgroep en de ongelijke ervaring van deze doelgroep met multi-touch interfaces, worden er vooral eisen gesteld aan de usability van de applicatie: hiermee worden de eisen bedoeld die invloed hebben hoe gebruikers de applicatie gebruiken en wat de gewenste ervaring van het gebruik is en niet de precieze werking van het te bouwen concept. Hieronder volgt een opsomming van deze eisen met argumentatie.

\begin{itemize}
	\item \textbf{Het doel en de wijze van gebruik van de muziektafel moet voor gebruikers op het eerste oog duidelijk zijn.} \\ Om mensen uit te nodigen de applicatie te gebruiken moet deze simpel in gebruik zijn en een vlakke leercurve hebben.
	\item \textbf{De muziektafel moet door 1 tot minstens 4 gebruikers te gebruiken zijn.} \\ Mensen kunnen er voor kiezen om met zijn vieren te beginnen met de applicatie, maar het moet ook mogelijk zijn om met minder te beginnen en voor mensen om zich later erbij aan te sluiten. Voor deze latere aansluiters helpt het om te zien dat andere mensen reeds bezig zijn met de tafel. Zo kunnen zij er zonder druk bij gaan staan en in hun eigen tempo de applicatie leren besturen, zonder dat zij andere spelers ophouden.
	\item \textbf{Het getoonde speelveld moet opgedeeld zijn in vier (4) gedeelten.} \\ Om mensen uit te nodigen mee te doen, zijn er te allen tijde vier speelvelden zichtbaar. Deze worden niet gebruikt indien er minder dan vier spelers zijn (tenzij \'e\'en speler meerdere speelvelden bedient), waardoor mensen zich eenvoudig in de spelbeleving kunnen mengen, zonder verdere stappen te moeten nemen.
	\item \textbf{Iedere speler heeft een eigen gedeelte van het speelveld tot zijn beschikking.} \\ Om ervoor te zorgen dat spelers elkaar niet in de weg zitten en iedereen op zijn eigen tempo kan werken met de applicatie, heeft iedere speler een eigen speelveld.
\end{itemize}

\section{Conceptueel Ontwerp}
\label{sec_concept}
In aanloop naar het ontwerp heeft de opdrachtgever vier voorstellen gekregen, verdeeld over twee categorie\"en: \'e\'en co\"operatief en drie competitief. Uit deze voorstellen is uiteindelijk een concept gekozen dat het beste aansloot bij het gedachtegoed van de opdrachtgever en het doel dat de opdrachtgever met de applicatie voor ogen had.

Het co\"operatieve voorstel bestond uit een muziektafel waarbij meerdere mensen samen werkten aan het componeren van een harmonieuze symfonie. Dit concept is geïnspireerd op een muziekspelletje op internet, waarbij met verschillende instrumenten een symfonie gecreëerd kan worden [http://games.co.za/music-sequencer.html]. De drie competitieve voorstellen waren spellen. De eerste was een boogschietspel, waarbij iedere speler moest voorkomen dat zijn doelwit geraakt werd door de pijlen van andere spelers af te weren, terwijl hij probeerde de doelwitten van de andere spelers te raken. De populariteit van spellen als Angry Birds [http://www.angrybirds.com/] diende hierbij ter inspiratie voor het mikken: de pijl en boog vormden hier een invulling van. Een ander voorstel was een multiplayer snake waarbij spelers elkaars slang moesten ontwijken, terwijl ze streefden naar het opeten van punten en bonussen. Ten slotte was er nog een voetbaltafel waarbij twee teams proberen de bal in het doel van het andere team te krijgen door middel van knoppen langs de zijkant die de poppetjes van hun team op het speelveld aansturen.

Vanwege het korte tijdsschema van dit project zijn deze voorstellen alleen kort, tekstueel uitgewerkt en voorzien van een ruwe schets. Er zijn geen functionele requirements voor elk voorstel opgesteld. De voorstellen zijn met een mondelinge toelichting overgedragen aan de opdrachtgever en uiteindelijk is in overleg besloten te kiezen voor de muziektafel. Dit concept had voor nieuwe gebruikers de laagste instapdrempel, liet het best de mogelijkheden van de multi-touch tafel zien en bevat een creatieve draai, waar de opdrachtgever in zijn missie naar streeft. [FIXME: motivatie]

\subsection{Eisen Muziektafel}
Voordat begonnen kon worden aan het functionele ontwerp van de muziektafel, moet eerst uitgezocht worden aan welke functionele eisen dit ontwerp zal moeten voldoen. Hieronder volgende de functionele eisen, waaraan het ontwerp, in combinatie met de geformuleerde eisen uit hoofdstuk~\ref{sec_domein}, moet voldoen. Indien een eis toelichting vereist, is deze gegeven, alsmede van wie de eis afkomstig is.

\begin{itemize}
	\item \textbf{Met ieder speelveld wordt een (1) muziekinstrument bespeeld.} \\ Door deze eis kunnen spelers doen waar zij het meeste zin in hebben, zonder gehinderd te zijn door het instrument dat andere spelers willen bespelen.
	\item \textbf{Een speler kan zelf kiezen welk instrument hij bedient met zijn speelveld.} \\ Dankzij deze eis kan iedere speler het voor hem meest interessante instrument bespelen, wat mensen met tal van verschillende, muzikale interesses uitnodigt gebruik te maken van de multi-touch tafel.
	\item \textbf{Een speler kan zelf bepalen met welk volume zijn instrument afgespeeld wordt.} \\ Om te voorkomen dat een speler afbraak doet aan de gecreëerde melodie, is het wenselijk als hij in dat geval het volume van zijn muziekinstrument kan aanpassen. Op deze wijze kan hij nog steeds spelen met de multi-touch tafel, zonder het idee te hebben andere spelers lastig te vallen. Deze eis volgt uit de wens om de tafel voor veel mensen, van iedere leeftijd, toegankelijk te maken.
	\item \textbf{Elk speelveld vertegenwoordigt één instrument. Door de aanraak-interface kunnen mensen bepalen van welk instrument op welk moment welke tonen afgespeeld worden.} \\ Deze eis volgt uit de wens van de opdrachtgever: spelers moet kennis maken met de multi-touch tafel.
	\item \textbf{De spelers moeten tegelijkertijd met minstens een (1) aanraking per speler tegelijk input kunnen geven aan de af te spelen noten.} \\ Ook deze eis volgt uit de wens van de opdrachtgever om de eigenschappen van de multi-touch tafel te etaleren: spelers moeten hun eigen speelveld kunnen bedienen, zonder dat de limieten van de tafel ze daarbij hindert. [FIXME: reference: The cueTable: Cooperative and Competitive Multi-Touch Interaction on a Tabletop]
	\item \textbf{De noten van de vier muziekinstrumenten worden gecombineerd tot een harmonieuze melodie.} \\ Deze eis volgt uit de wens van de opdrachtgever om mensen aan te zetten tot het gebruik: des te meer mensen meehelpen aan het componeren van een melodie, des te beter deze wordt.
	\item \textbf{Het mag niet mogelijk zijn dat er een valse melodie gecreëerd wordt.} \\ Als de gecomineerde melodie prettig in het gehoor ligt, zijn gebruikers sneller geneigd om door te gaan met het componeren ervan. [citation needed]
	\item \textbf{De af te spelen noten zijn per 16e maat in te stellen.} \\ De noten zijn per 16e maat instelbaar. Het is nog onduidelijk wat voor het componeren van muziek uiteindelijk wenselijker is: indien een noot twee 16e maten achter elkaar aanstaan, wordt deze dan gedurende deze twee 16e maten continu afgespeeld, of wordt de noot de tweede 16e maat opnieuw aangeslagen.
	\item \textbf{Per 16e maat zijn nul of meer noten te selecteren die afgespeeld moeten worden.} \\ Hierdoor is het mogelijk om per muziekinstrument zowel harmonie als melodie te krijgen.
	\item \textbf{Per muziekinstrument kunnen tien (10) toonhoogten en 16 16e maten geselecteerd worden.} \\ Deze eis volgt uit het niet mogen componeren van een valse melodie. Door per octaaf de noten \textit{A, B, D, E en F\#} te gebruiken, kan geen valse melodie gecreëerd worden. [citation needed] Om ervaren gebruikers de mogelijkheid te geven om complexere melodie\"en te componeren, hebben ze de beschikking over twee octaven.
	\item \textbf{Als er langer dan een minuut geen gebruik wordt gemaakt van het programma, wordt een demomodus afgespeeld, waarbij een voorgeprogrammeerde melodie wordt opgebouwd en afgespeeld. Zodra een speler het scherm aanraakt, wordt of de eerder gebouwde melodie op ieders speelveld teruggezet of wordt een leeg speelveld getoond.} \\ Deze eis volgt uit de wens om de applicatie uitnodigend te maken. Door deze demo-modus wordt voor de gebruiker gedemonstreerd wat er met de applicatie kan en hoe deze met behulp van de multi-touch tafel bediend kan worden.
	\item \textbf{De muziek wordt standaard afgespeeld met een snelheid van 120 beats per minuut (BPM), wanneer dit niet instelbaar is.} \\ Deze eis zorgt ervoor dat er een redelijke tempo in de muziek zit, zodat een simpele gecre\"eerde melodie al spannend genoeg is om aan verder te werken. [citation needed]
	\item \textbf{Tijdens het afspelen wordt de gecreëerde muziek gevisualiseerd.} \\ Dit is wederom voor het uitnodigende karakter van de applicatie gedaan, om naast de ge\"interesseerden in audio ook de visueel ingestelde mensen aan te trekken.
	\item \textbf{Nadat de laatste kwartmaat afgespeeld is, wordt opnieuw begonnen bij de eerste kwartmaat.} \\ Om te voorkomen dat spelers steeds de melodie opnieuw moeten starten en om de bediening zo eenvoudig mogelijk te houden, is het afspelen van de melodie een doorlopend proces: men kan zo aanschuiven, de noten van zijn instrument instellen en meteen het resultaat horen.
	\item \textbf{De door de spelers geselecteerde noten blijven geselecteerd wanneer opnieuw bij de eerste kwartmaat begonnen wordt.} \\ Ook dit is onderdeel van de tafel uitnodigend maken en spelers het idee te geven dat ze iets aan het bouwen zijn: hun melodie blijft bestaan en verbeterd worden.
	\item \textbf{Het speelveld van een speler moet door de speler in \'e\'en handeling ontdaan kunnen worden van alle geselecteerde noten.}
\end{itemize}

\subsection{Uitwerking Muziektafel}
Op basis van deze eisen is een conceptueel ontwerp gemaakt. Deze schets is opgenomen in Figuur~\ref{fig_muziektafel}. Te zien hierin is allereerst de cirkel waarop spelers noten kunnen selecteren: de notenbalk. Het onderscheid tussen de verschillende speelvelden is aangegeven door vier verschillende kleuren. Op de cirkel liggen 10 toonhoogtes en 16 16e maten. Standaard is het tempo ingesteld op 120 beats per minuut, maar dit is door middel van de plus- en minknop in het midden van de cirkel in te stellen.

Onder ieder speelveld staan twee knoppen in dezelfde kleur als het speelveld, zodat voor de spelers duidelijk is welke knoppen van hen zijn. Met de linkerknop worden alle geselecteerde noten van het speelveld verwijderd. Om aan te geven dat deze knop iets verwijderd, is gebruik gemaakt van een vuilnisbak-icoon. Met de rechterknop hebben spelers de kans om het instrument van hun speelveld aan te passen. Om aan te geven dat het hun instrumentkeuze betreft, wordt er op de knop een icoon getoond van het instrument waar ze momenteel mee spelen. Wanneer op de instrumentkeuzeknop gedrukt wordt, wordt er een overlay getoond over de notenbalk. In deze overlay kan de speler een van de acht beschikbare instrumenten kiezen: een piano, gitaar, basgitaar, drumstel, fluit, viool, orchel of triangel. In Figuur~\ref{fig_muziektafel} is deze overlay voor het blauwe speelveld zichtbaar. Wanneer de speler op een instrument klikt, sluit de overlay zich en zal de notenbalk van dat speelveld afgespeeld worden met het gekozen instrument.

Daarnaast is er boven de notenbalk een extra onderdeel zichtbaar: de alternatieve notenbalk. Iedere speler heeft de beschikking over 2 sets van 16 16e maten. Een van deze twee wordt weergegeven op de notenbalk en kan door de speler bewerkt worden, de ander wordt boven de notenbalk weergegeven als knop. Wanneer op die knop gedrukt wordt, worden de twee sets noten omgewisseld; de set noten van de notenbalk wordt vervangen door de andere set. Bij het afspelen van de noten wordt vanaf de eerstvolgende 16e maat de nieuwe set noten gebruikt. Wanneer de wis-knop linksonder op het speelveld gebruikt wordt, wordt alleen de set noten gewist die op dat moment in de notenbalk staat. De andere set blijft bewaard.

Zodra de applicatie gestart wordt, wordt begonnen met het afspelen van de notenbalk. De noten van de vier speelvelden worden per 16e maat tegelijk afgespeeld. Om aan te geven welke 16e maat afgespeeld wordt, wordt de notenbalk voor deze 16e maat gearceerd. In Figuur~\ref{fig_muziektafel} is dat te zien aan de feller gekleurde kolommen.

\subsection{Evaluatie LoFi-prototype}
Om uit te vinden hoe intuïtief het bedachte ontwerp voor de gewenste doelgroep is, is de schets uit het verslag voorgelegd aan een aantal mensen. Ook is een eerste versie van de applicatie gemaakt, die alleen een werkende notenbalk-cirkel bevat. Door middel van de muis kunnen noten aangeklikt worden en kan de applicatie hiermee gebruikt worden door één persoon.

Vanwege het gewenste uitnodigende karakter van de multi-touch tafel is gekozen om bij de evaluatie voornamelijk de intuïtieve werking van het ontwerp te testen. We hebben hierbij het volgende script gebruikt om de gebruikers te ondervragen:

\begin{itemize}
  \item De proefpersoon krijgt de schets van het ontwerp uit Figuur~\ref{fig_muziektafel} te zien.
  \item \textbf{Vraag 1:} Als je naar de interface kijkt, wat denk je dat de applicatie doet?
  \item \textbf{Vraag 2:} Hoe zou je deze besturen? Wat zou je als eerste proberen?
  \item Na deze twee vragen wordt de proefpersoon uitgelegd dat het een muziekapplicatie betreft.
  \item \textbf{Vraag 3:} Nu je weet wat de applicatie doet, hoe zou je het besturen?
  \item \textbf{Vraag 4:} Voor hoeveel spelers is het spel geschikt?
  \item Indien de speler niet bedacht heeft dat het een multi-touch applicatie is, wordt hem de besturing van de notenbalk uitgelegd. Ook wordt er verteld dat het op een multi-touch tafel gespeeld wordt samen met anderen.
  \item \textbf{Vraag 5:} Vind je de representatie van noten duidelijk?
  \item \textbf{Vraag 6:} Als je het muziekinstrument wilt veranderen, hoe zou je dat doen?
  \item \textbf{Vraag 7:} Als je het tempo van de muziek wilt aanpassen, hoe zou je dat doen?
  \item \textbf{Vraag 8:} Waar denk je dat het extra vlak boven de noten voor bedoeld is?
  \item \textbf{Vraag 9:} Als je opnieuw wilt beginnen met een leeg veld, hoe zou je dat doen?
  \item \textbf{Vraag 10:} Heb je verder nog opmerkingen over het doel van de applicatie en hoe de interface er nu uit ziet?
\end{itemize}

De eerste twee vragen dienen om te achterhalen of de applicatie de juiste indruk geeft bij mensen: nodigt de applicatie uit tot muziek maken? De derde en vierde vraag zijn bedoeld om te controleren of de interface laat zien het door meerdere mensen tegelijk bestuurd kan worden en of hoe de gebruiker zelf denkt dat dit werkt. Vervolgens worden een voor een naar de intuïtiviteit van de overige features gevraagd. Indien de gebruiker het antwoord op een vraag niet wist, werd hem dit uitgelegd. Bij de laatste vraag kon de gebruiker alle feedback die hij verder nog had kwijt.

\subsubsection{Resultaten}



\begin{figure}
  \includegraphics[width=84mm]{lofi-prototype.jpg}
  \caption{Schets van de interface van de muziektafel}
  \label{fig_muziektafel}
\end{figure}


\section{Gedetailleerd Ontwerp}
\label{sec_detail}


\section{Evaluatie}
\label{sec_evaluatie}

\section{Discussie}
\label{sec_discussie}

\section{Conclusie}
\label{sec_conclusie}

\section{Acknowledgments}
%Bedankt Andries, Matthijs en Thomas van KITT


\end{document}
